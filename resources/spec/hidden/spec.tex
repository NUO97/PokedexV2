\documentclass[programming]{../../../../mfcs}
\newcommand{\projectduedate}{Wednesday, May 31st}
\newcommand{\quarter}{Spring 2017}
\usepackage{color}
\usepackage{multicol}
\usepackage{wrapfig}
\setlength{\belowcaptionskip}{-30pt}
\setlength{\intextsep}{-20pt}
\setlength{\textfloatsep}{0pt}

\course{CSE 154}{Web Programming}{\quarter}
\NoDate
\topic{Homework Assignment 7: Pokedex V2}
\usepackage{csquotes}
\usepackage{tabularx}

\begin{document}
\vspace{-3.8em}

\hfill\begin{varwidth}{0.5\textwidth}
{\large {\bf\color{colour} Due Date:} \projectduedate}
\end{varwidth}
\vspace{2em}

This assignment is about using PHP together with SQL to create, modify, and query information in a database.

\vspace{2em}
\begin{question}{Overview}
  You are provided the following files:
  \begin{itemize}
    \item \texttt{main.html} - the main page of the application, which lets a user choose to start a
      game or trade Pokemon with another user.
    \item \texttt{main.js} - the JS for \texttt{main.html}
    \item \texttt{pokedex.html} - the pokedex/game view of the application, which lets a user choose
      a Pokemon to play with and then play a Pokemon card game with another player.
    \item \texttt{pokedex.js} - the JS for \texttt{pokedex.html}
    \item \texttt{styles.css} - the styles for \texttt{main.html} and \texttt{pokedex.html}
  \end{itemize}

  You will turn in the following files:
  \begin{itemize}
    \item \texttt{setup.sql} - a small SQL file that sets up your personal Pokedex table.
    \item \texttt{getcreds.php} - a web service for retrieving your player credentials (PID and token).
    \item \texttt{select.php} - a web service for adding a Pokemon to your Pokedex table.
    \item \texttt{insert.php} - a web service for adding a Pokemon to your Pokedex table.
    \item \texttt{update.php} - a web service for naming a Pokemon in your Pokedex.
    \item \texttt{delete.php} - a web service for removing a Pokemon from your Pokedex table.
    \item \texttt{trade.php} - a web service for updating your Pokedex list after a Pokemon ``trade''.
    \item \texttt{common.php} - shared PHP functions for your other PHP files.
  \end{itemize}
\end{question}


\begin{question}{Behavior}
  \vspace{1em}
  \subquestion*{Creating Your Own Database - \texttt{setup.sql}}
  Before starting the PHP files, you will need to set up your own SQL database. In Cloud 9, you can
  do so by starting the mysql terminal and entering the following commands: 
  \begin{verbatim}
CREATE DATABASE hw7;
use hw7;\end{verbatim}
  This is the database you'll be querying into throughout the assignment. The provided front-end code will use your
  web services and the data in your database to keep track of which Pokemon you have caught. Storing your pokemon in
  a database (instead of in a JS array as suggested in HW5) allows us to maintain the state after refreshing or exiting the web page.
  \newline

  Write a SQL file that creates a table called ``Pokedex'' to store your collected Pokemon. This table
  should have three columns: ``name'' for each Pokemon's name which also serves as the table's
  \texttt{PRIMARY KEY} (e.g., ``Bulbasaur''), ``nickname'' for
  each Pokemon's nickname (e.g., ``Bulby''), and ``datefound'' for the date and time you collected the Pokemon. 
  The name and nickname columns should have
  \texttt{VARCHAR} data types and allow strings of at least 20 characters in length. 
  \newline
  
  We can represent date and time in mysql using the \texttt{DATETIME} data type. This type represents a date and time
  in the format \texttt{YYYY-MM-DD HH:MI:SS} (e.g., 2017-05-15 13:54:00 to represent 1:54 PM on May 15th, 2017).
  \newline

  Your database name (hw7), table name (Pokedex), column names (name, nickname, datefound) must match exactly those
  here in the spec. Your columns must match exactly those described here in the spec.
  \newline

  Note the following SQL commands that will probably prove useful:
  \begin{verbatim}
SOURCE setup.sql;                 -- runs your setup.sql file
SHOW databases;                   -- lists databases in your mysql
USE hw7;                          -- tells mysql to use your hw7 database
SHOW tables;                      -- list tables in your currently active (hw7) database
DESCRIBE <tablename>;             -- gives information about the columns of a table
DROP TABLE <tablename>;           -- careful with this one, it deletes a table entirely\end{verbatim}

  \subquestion*{Fetching Player Credentials - \texttt{getcreds.php}}
  This PHP file returns the user's player ID (PID) and token. For this assignment, your PID
  will be your UW netid. These PID and token values will be used by the front end code and the games webservice for
  verifying that players are who they say they are when they play moves in battle mode, and trade with one another.
  \newline

  You will need to generate your token to play games and trade with other students on our server. To do so, visit
  \url{https://webster.cs.washington.edu/pokedex-2/uwnetid/generate-token.php}. The PID and token
  values displayed should be carefully copy/pasted in your \texttt{getcreds.php} file.
  \newline

  In this PHP file, you should print as a \texttt{plain text} response your PID followed by your
  token, each on their own line. Note that there are no query parameters for this file, so you
  print these values whenever the web service is called. Below is an example response for Whitaker's credentials:
\begin{verbatim}
whitab
poketoken_123456789.987654321\end{verbatim}

  \subquestion*{Fetching Pokedex Data - \texttt{select.php}}
  \texttt{select.php} should output a JSON response of all Pokemon you have found in your Pokedex, 
  including the name, nickname, and found date/time for each Pokemon. Below is an example response:
\begin{verbatim}
{
  "pokemon": [
     { "name" : "Bulbasaur",
       "nickname" : "Bulby",
       "datefound" : "2017-05-15 13:54:00" },
     { "name" : "Charmander",
       "nickname" : "Charmy",
       "datefound" : "2017-05-16 08:45:10" },
       ... ]
}\end{verbatim}
This PHP web service should take no query parameters.
\newpage
  \subquestion*{Adding a Pokemon to your Pokedex - \texttt{insert.php}}
  Query Parameters (\texttt{POST}):
  \begin{itemize}
    \item \texttt{name} - name of Pokemon to add
    \item \texttt{nickname} (optional) - nickname of added Pokemon
  \end{itemize}
  \texttt{insert.php} adds a Pokemon to your Pokedex table, given a required \texttt{name}
  parameter. The \texttt{name} should be added to your Pokedex in the format given in the request (in
  other words, you should not modify the casing/spacing/special characters for the name).
  \newline

  If passed a \texttt{nickname} parameter, this nickname should also be added with the Pokemon. 
  Otherwise, the nickname for the
  Pokemon in your Pokedex table should be set to the Pokemon's name in all uppercase (e.g.,
  ``BULBASAUR'' for ``Bulbasaur''). You should also make sure to
  include the date/time you added the Pokemon. In PHP, you can get the current date-time in the
  format for the previously-described SQL \texttt{DATETIME} data type using the following code:
  \begin{verbatim}
date_default_timezone_set('America/Los_Angeles');
$time = date('y-m-d H:i:s');\end{verbatim}
  You should add the result \texttt{\$time} variable to add to your \texttt{datefound} table column.
  \newline

  Upon success, you should output a JSON result in the format:
  \begin{verbatim}
{ "success" : "Success! <pokemon> added to your Pokedex!" }\end{verbatim}
  where you should replace \texttt{<pokemon>} with the Pokemon's name in the same format it has in
  the table. If the Pokemon is
  already in the Pokedex (as determined by a duplicate name field), you should print a message with
  a 400 error header in
  the JSON format: 
  \begin{verbatim}{ "error" : "Error: Pokemon <pokemon> already found." }\end{verbatim}
  You should not change anything in your
  Pokedex as a result. 

  \subquestion*{Removing a Pokemon from your Pokedex - \texttt{delete.php}}
  Query Parameters (\texttt{POST}):
  \begin{itemize}
    \item \texttt{name} - name of Pokemon to remove, or
    \item \texttt{mode=removeall} - removes all Pokemon from your Pokedex 
  \end{itemize}
  If passed \texttt{name}, \texttt{delete.php} removes the Pokemon with the given name
  (case-insenstive) from your Pokedex. For example, if you have a Charmander in your Pokedex table
  and a request to \texttt{delete.php} with \texttt{name} passed as \texttt{charMANDER} is made,
  your Charmander should be removed from your table.
  \newline

  Upon success in this case, you should print a JSON result in the format:
  \begin{verbatim}
{ "success" : "Success! <pokemon> removed from your Pokedex!" }\end{verbatim}
  (again, replacing \texttt{<pokemon>} with the removed Pokemon's name).
  \newline

  If passed a Pokemon name that is not in your Pokedex, you should print a message with a 400 error
  header in the JSON format: 
  \begin{verbatim}{ "error" : "Error: Pokemon <pokemon> not found in your Pokedex." }\end{verbatim}
   where \texttt{<pokemon>} is replaced with the value of the passed \texttt{name} (maintaining
  letter-casing). Your table should then not change as a result.
  \newline

  Otherwise if passed \texttt{mode=removeall} all Pokemon should be removed from your Pokedex table.
  \newpage

  Upon success in this case, you should print a JSON result in the format:
  \begin{verbatim}
{ "success" : "Success! All Pokemon removed from your Pokedex!" }\end{verbatim}

  If passed a mode other than \texttt{removeall}, you should print a message with a 400 error header
  in the format: 
  \begin{verbatim}{ "error" : "Error: Unknown mode <mode>." }\end{verbatim}
  where \texttt{mode} is replaced with whatever
  value the user passed for this query parameter. If passed both \texttt{mode=removeall} and
  \texttt{name} as query parameters, the \texttt{name} functionality should hold precedence (in
  other words, remove only the Pokemon with the given name and don't execute the removeall
  behavior).

  \subquestion*{Trading Pokemon - \texttt{trade.php}}
  Query Parameters (\texttt{POST}):
  \begin{itemize}
    \item \texttt{mypokemon} - name of Pokemon to give up in trade
    \item \texttt{theirpokemon} - name of Pokemon to receive in trade
  \end{itemize}
  \texttt{trade.php} takes a Pokemon to remove from your Pokedex
  \texttt{mypokemon} (case-insensitive) and a Pokemon to add to your Pokedex \texttt{theirpokemon}. 
  \newline

  This web service add should insert \texttt{theirpokemon} into your Pokedex and remove
  \texttt{mypokemon} from your Pokedex. Upon success, you should print a JSON result in the format: 
  \begin{verbatim}
{ "success" : "Success! You have traded your <mypokemon> for <theirpokemon>!" }\end{verbatim}
  You should give the added Pokemon the default nickname specified in \texttt{insert.php}. If your you do not have the passed \texttt{mypokemon} in your Pokedex table, you should print a
  400 error header with a message in the same format as the error message specified in
  \texttt{delete.php} (using \texttt{mypokemon} for the Pokemon's name).
  \newline

  Otherwise, if you already have the passed \texttt{theirpokemon} in your Pokedex, you should print
  a 400 error header with a message in the JSON format: 
  \begin{verbatim}
{ "error" : "Error: You have already found <theirpokemon>." }\end{verbatim}

  If either error occurs, your table should not be changed as a result.

  \subquestion*{Renaming a Pokemon in your Pokedex - \texttt{update.php}}
  Query Parameters (\texttt{POST}):
  \begin{itemize}
    \item \texttt{name} - name of Pokemon to rename
    \item \texttt{nickname} - new nickname to give to Pokemon
  \end{itemize}
  \texttt{update.php} updates a Pokemon in your Pokedex table with the given \texttt{name}
  (case-insensitive) 
  parameter to have the given \texttt{nickname} (overwriting any previous nicknames). 
  \newline

  Upon success, you should print a JSON result in the format: 
  \begin{verbatim}
{ "success" : "Success! Your <name> is now named <nickname>!" }\end{verbatim}
  As in the previous files, \texttt{name} should be printed in the same format as it is
  in the Pokedex table.
  \newline

  If you do not have the Pokemon with the passed \texttt{name} in your Pokedex, you should output
  the error behavior as in the same case for \texttt{delete.php}.

  \subquestion*{\texttt{common.php}}
  You should factor any shared code into \texttt{common.php} and turn it in with the rest of your
  PHP files. Recall that you can use \texttt{include 'common.php'} at the top of a PHP file to
  include all functions that are found in a file called \texttt{common.php} (requiring it is in the
  same directory as the file including it).
  \newline

  For any PHP web service with \texttt{GET} or \texttt{POST} parameters, if the user does not
  provide a required parameter, a 400 error message should be output in the JSON format:
  \begin{verbatim}
{ "error" : "Missing <parametername> parameter"}\end{verbatim}
or
  \begin{verbatim}
{ "error" : "Missing <parameter1> and <parameter2> parameters"}\end{verbatim}
if there are two required parameters missing, or
  \begin{verbatim}
{ "error" : "Missing <parameter1> or <parameter2> parameters"}\end{verbatim}
if both parameters are missing for a web service that requires one or the other (as is the case with
\texttt{delete.php}). These error responses should take precedence over any other error for each web service.

\end{question}

\begin{question}{Development Strategy}
  \subquestion*{SQL}
    This homework should give you a lot of experience using the \texttt{mysql} program to keep track of what changes
    are being made to your database.
     \begin{itemize}
       \item Run basic versions of your queries from the \texttt{mysql} terminal before putting them
         into your PHP.
       \item Use \texttt{try/catch(PDOException \$pdoex)} to trap SQL exceptions in your PHP code,
         and print them for debugging.
     \end{itemize}

  \subquestion*{PHP}
    The provided front end for this homework is NOT a good testing program. It assumes that your code works, and makes
    many calls against your code in quick succession. We STRONGLY encourage you to call your PHP functions over the web
    before trying to use your code in concert with the provided front end.
    \newline

    For \texttt{GET} requests (\texttt{getcreds.php} and \texttt{select.php}) the easiest thing to do is simply use a browser to visit the URL and pass the query params.
    \newline

    For the other PHP files that you implement as \texttt{POST} requests, you'll need to do something a little bit more
    complicated. This is because it's harder to simulate \texttt{POST} requests than \texttt{GET} requests, but you have some options:

     \begin{itemize}
       \item Make a dummy HTML page that lets you write JS AjaxPostPromises, or \texttt{POST}
         XMLHttpRequests from the JS console.
       \item Make a dummy HTML page with a form that submits to your PHP program.
       \item Use a program like Postman \url{https://www.getpostman.com/} to craft \texttt{POST} requests against your API.
             (Note: you need Postman Interceptor if you are developing on Cloud9 -- this lets Postman use your browser
             login information to authenticate your request to Cloud9).
       \item One other way is to test with GET, and change to \texttt{POST} after you are satisfied that it works. However, you
             should still test that the \texttt{POST} works before you turn your homework in, and for this reason, we encourage you to use
             another testing strategy to get into the flow of actually testing \texttt{POST}s.
     \end{itemize}

  \subquestion*{General}
     \begin{itemize}
       \item Get your database setup, implement \texttt{setup.sql} and practice making some database SELECT, INSERT, UPDATE, DELETE queries from the \texttt{mysql} terminal
       \item Implement \texttt{getcreds.php} to get going on the PHP part of the assignment.
       \item Be on the lookout for common code to factor into \texttt{common.php}
       \item Implement \texttt{select.php} using data that you have manually inserted into the DB from the \texttt{mysql} terminal
       \item Implement \texttt{insert.php}, and verify that it works first in the database, and then with \texttt{select.php}
       \item Implement \texttt{update.php} and \texttt{delete.php}
       \item Implement \texttt{trade.php}
     \end{itemize}
\end{question}


\begin{question}{Implementation and Grading}
For full credit, your SQL and PHP code should follow the rules listed in the Style Guide on the course web site.
\newline

Your PHP code should not cause errors or warnings. Do not use the \texttt{global} keyword. Use good indentation/spacing,
and avoid long lines over 100 characters. Avoid redundant code, and use parameters and return values properly.
Capture common operations as functions to keep code size and complexity from growing.
\newline

Additionally, for this assignment, your PHP code must pass a PHP strict standards check. To check for this, you
should execute the following line of code at the beginning of every PHP file that you write:
  \begin{verbatim}
error_reporting(E_ALL); \end{verbatim}

Note that it might not make sense for this line of code to be physically present in every PHP file.
\newline

Format your PHP and SQL code similarly to the examples from class. Properly use whitespace and indentation.
\newline

Do not place your solution on a public web site. Submit your own work and follow the course misconduct policy.
\end{question}
\end{document}
